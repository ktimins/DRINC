{
    \renewcommand*{\theenumi}{\thesubsection.\arabic{enumi}}
    \renewcommand*{\theenumii}{\theenumi.\arabic{enumii}}
    \renewcommand*{\theenumiii}{\theenumii.\arabic{enumiii}}

    \section{Physical Architecture}

    \subsection{Back End Requirements}
            
    \subsubsection{The Raspberry Pi}
        \begin{description}
	        \item[Processor] Broadcom 700 MHz
	        \item[RAM]       256MB
	        \item[Graphics]  VideoCore IV
	        \item[OS]        Debian Linux
        \end{description}

    \subsection{Front End Requirements}

    \subsubsection{The Nexus 7 Android}
        \begin{description}
	        \item[Processor] ARM Cortex-A9 Nvideia Tegra 3 T30L 1.2 GHz quad-core
	        \item[RAM]       1 GB
	        \item[Graphics]  7in Touch screen
	        \item[OS]        Android Jelly Bean
        \end{description}
    
    \subsection{Hardware Design Overview}

    Touch screen talks to backend server on the raspberry pi. This formats the 
    proper protocol commands that are sent to the arduino. The arduino then 
    processes these commands and sends the proper data instructions to the 
    proper pins that will control the solenoids.

    \subsubsection{Front End}
        \begin{itemize}
            \item The front end is an Android device which will run the front 
            end software described later.
            \item The front end will connect to the back end with a custom 
            protocol that will communicate with the front end over a USB cable.
        \end{itemize}

    \subsubsection{Back End}
        \begin{itemize}
            \item The backend is a Raspberry PI running the backend software 
            and an Arduino device that will communicate with the valves/flow
            meters directly.
        \end{itemize}


    
    \subsubsection{Frame}

	    The frame of the machine will accommodate all other components.  It 
        will suspend 9 bottles 750mL in volume above a track consisting of a 
        cup holder moving horizontally along two rails assisted by a pair of 
        servos.  The frame will also house an Arduino Mega, a Raspberry Pi, 
        and a standard ATX power supply, as well as a mounting point for the 
        frontend device.

    \subsubsection{Electronics}

        The backend will communicate with an Arduino Mega microcontroller.  
        The microcontroller will in turn control a valve attached to each 
        available mixer.  Additionally, the microcontroller will control a 
        pair of servos to move the cup holder to any of the available mixers.

	    \paragraph{Arduino Mega}
        \begin{itemize}
		    \item 16MHz core clock
		    \item-124KB available flash
		    \item-49 digital I/O pins
		    \item-15 digital I/O pins capable of PWM
		    \item-16 analog I/O pins
        \end{itemize}

	    \paragraph{Valves}
        \begin{itemize}
		    \item Aquatech AQT15SP Solenoid Valve
		    \item 12V operating voltage
		    \item $\frac{3}{4}$in diameter
        \end{itemize}

	    \paragraph{Servos}
        \begin{itemize}
		    \item 43R Robot Rotation Servo
		    \item 5V operating voltage
		    \item Direction and speed controlled
		    \item 60 RPM
        \end{itemize}

        \paragraph{Power Supply} 
	    A standard 400W ATX power supply will adequately power all 
        electronic components. The backend can bring the PSU out of standby 
        and shut it back down when the machine is not in use.

}
                
