\chapter{Hardware Design Overview}
{
    \renewcommand*{\theenumi}{\thesubsection.\arabic{enumi}}
    \renewcommand*{\theenumii}{\theenumi.\arabic{enumii}}
    \renewcommand*{\theenumiii}{\theenumii.\arabic{enumiii}}


    \section{Description}
        The DRINC project's hardware consists of all physical devices used 
        to transport, pour, and support drinks.
        \begin{itemize}
            \item The frame (made up of the following subcomponents):
            \begin{itemize}
                \item The track
                \item The valves
                \item The support structure
            \end{itemize}
            \item The PSU
            \item The Raspberry Pi server
            \item The Arduino microcontroller
            \item The Nexus 7 frontend
        \end{itemize}

    \section{Support Structure}
        DRINC will be held in a frame constructed from high quality ABS 
        plastic capable of supporting a minimum of 9 liters of liquid in 
        bottles.  It will also have space to house the drink transportation 
        track, the backend server, the Arduino, and the PSU.

    \section{Track}
        The track is made up of a pair of toothed rails running the length 
        of the frame attached to rotation controlled servos.  The track will 
        hold a cup of varying size, as well as detect whether or not a cup 
        is in place.

    \section{Valves}
        Each mixer be equipped with a custom built valve consisting of a 
        shaped tube connected to a position controlled servo.  The Arduino 
        microcode will control each valve.


    \section{PSU}
        DRINC will be powered by a 300W ATX PSU.  This will be enough to 
        power all valves, the track, the backend server, and the 
        microcontroller.

    \section{Arduino} 
        An Arduino Mega256 microcontroller will be used to control all track 
        and valve functions required by DRINC.  The microcontroller will 
        draw power from the DRINC PSU.  The microcode running on the is 
        structured as follows:
        \begin{itemize}
            \item The microcontroller turns on and initializes itself
            \item Wait for a serial data packet from the backend
            \item Upon receiving all serial data, check to make sure it is a valid drink recipe
            \item If drink is valid, move track to first valve in recipe
            \item Calculate time valve must be open based on recipe
            \item Open valve, then close it after calculated time
            \item Repeat process for remaining valves in recipe
            \item Move track to finished drink position
            \item Wait until cup is removed and replaced
            \item Send backend a ready signal
            \item Return to wait for data state
        \end{itemize}

}
