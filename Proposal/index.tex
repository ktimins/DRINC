\documentclass[letterpaper]{article}
\usepackage{graphicx}
\usepackage{underscore}
\usepackage{enumitem}
\usepackage{float}
\usepackage{fancyvrb}
\usepackage{color}
\usepackage{acronym}
\usepackage[hypertexnames=false, bookmarks=true, bookmarksnumbered=true,	breaklinks=true, linkbordercolor={0 0 1}]{hyperref}
\usepackage{xcolor}
\usepackage{parskip}
\usepackage{draftwatermark}
\usepackage{fancyhdr}
\renewcommand{\footrulewidth}{0.4pt}
\usepackage{lastpage}
\usepackage[bookmarks=true]{hyperref}
\hypersetup{
    bookmarks=false,    % show bookmarks bar?
    pdfkeywords={TeX, LaTeX, graphics, images}, % list of keywords
    colorlinks=true,       % false: boxed links; true: colored links
    linkcolor=blue,       % color of internal links
    citecolor=black,       % color of links to bibliography
    filecolor=black,        % color of file links
    urlcolor=blue,        % color of external links
    linktoc=page            % only page is linked
}%

%% Make the link look like web hyperlinks
\makeatletter
\Hy@AtBeginDocument{%
  \def\@pdfborder{0 0 1}% Overrides border definition set with colorlinks=true
  \def\@pdfborderstyle{/S/U/W 1}% Overrides border style set with colorlinks=true
  %% Hyperlink border style will be underline of width 1pt
}
\makeatother
\hypersetup{%
  colorlinks=true,
  linkcolor=blue,
  linkbordercolor=blue,
  pdfborderstyle={/S/U/W 1}
}

%% Define a new 'leo' style for the package that will use a smaller font.
\makeatletter
\def\url@leostyle{%
\@ifundefined{selectfont}{\def\UrlFont{\sf}}{\def\UrlFont{\small\ttfamily}}}
\makeatother
%% Now actually use the newly defined style.
\urlstyle{leo}

\newcommand{\theTitle}{Dynamically Refreshing Interplexing Number of Cordials}
\newcommand{\subTitle}{DRINC}
\newcommand{\class}{Senior Projects}
\newcommand{\fNames}{Brandon Arnold, Hoang Phan, Owen Ledvina, Kyle Timins}
\newcommand{\lNames}{Arnold, Phan, Ledvina, Timins}

\pagestyle{fancy}
\fancyhf{} % remove everything
  \lhead{\theTitle}
  \chead{}
  \rhead{\subTitle}
\lfoot{\class}
\cfoot{\lNames}
\rfoot{\thepage\ of \pageref*{LastPage}}
\fancypagestyle{plain}{%
  \fancyhf{} % clear all header and footer fields
%  \lhead{\theTitle}
%  \chead{}
%  \rhead{\leftmark}
  \lfoot{\class}
  \cfoot{\lNames}
  \rfoot{\thepage\ of \pageref*{LastPage}}
}

\begin{document}
\thispagestyle{plain}
\begin{center}
{\LARGE \textbf{\theTitle}}\\\vspace{0.5cm}
{\Large \textbf{Proposal}}\\\vspace{0.5cm}
by\\
{\large \fNames}

\end{center}

\section{Broad Scale Description}

The \theTitle{} (\subTitle{}) is a machine that can create drinks from
multiple bottles or sources. Such a machine has the potential to produce
drinks with higher accuracy of mixture proportions.\footnote{Humans use
methods like seconds poured or pour equivalent heights in a glass, but 
none actually measure the volume that has flown through the spout.} Along
with this, humans tend to have differing tastes and likes. This means
that the recipe for a drink could vary between customer to customer.
To reach the desired potential for the \subTitle{} machine, it must be 
able to automate the creation of the drink and be able to accept any recipe
that uses the ingredients, and quantities, it contains. While there are
multiple drink pouring machines available on the internet ~\cite{BaR2D2}
~\cite{Inebriator}, the \subTitle{} will allow for easier automation and
use.


\section{More Detail} %% Someone should change this section title

The \subTitle{} will have an array of multiple bottles. This allows for
a massive variety of mixed drinks. To be able to handle this, the machine
will handle this task by having a database containing a list of drinks. 
To be able to measure the amount of each liquid in the mixtures, the 
machine will measure the flow of the liquid. This will also allow the 
\subTitle{} to keep track of how much of each liquid is left, at any given
time.

The \subTitle{} will have two front ends: one via a web site and one via an
Android tablet. The web site will allow the owner to administer the machine.
The Android tablet will be attached to the machine. This will be the main
front-end used by the user to select a drink to pour.

\section{Deliverables}

The following are deliverables for this project:
\begin{itemize}
    \item Grid array holding all bottles with attached flow meters
    \item Micro computer controlling the movement of cup on grid
    \item Micro computer controlling the flow of the drinks
    \item Web based front end for drink creation per user and 
          administration
    \item Android tablet front end on robot for use front end    \end{itemize}

Some possible deliverables for this project:
\begin{itemize}
    \item NFC relating to drinkers profile
    \item LED light show while pouring drinks
\end{itemize}    

\bibliography{ref}
\bibliographystyle{plain}
\end{document}